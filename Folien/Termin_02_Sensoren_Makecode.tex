\documentclass{beamer}

\usepackage[utf8]{inputenc}
\usepackage{hyperref}
\hypersetup{
    colorlinks = true
    }
\usepackage{graphicx}

%Information to be included in the title page:
\title{GTA Einführung Robotik mit LEGO Mindstorms}
\author{Mattias Schlenker}
\institute{Wilhelm-Ostwald-Gymnasium}
\date{19. November 2020}

\begin{document}

\frame{\titlepage}

\begin{frame}
\frametitle{Welche Sensoren kennt Ihr?}
\begin{itemize}
\item Platzhalter für Eure Hausaufgabe!
\item Herr Schlenker macht dann Copy and Paste aus dem Chat
\end{itemize}
\end{frame}


\begin{frame}
\frametitle{LEGO-Sensoren}
\begin{itemize}
\item  Entfernungssensor
\item  Farbsensor
% \item   Schwarz-Weiss-Sensor $\Rightarrow$ Line-Follower
% \item   Magnetometer
\item  interne Sensoren (Drehwinkel der Räder oder Poti eines Servos)
\item (Temperatur)
% \item Gassensoren
% \item Radar, LIDAR, Kameras
\end{itemize}

Damit müssen wir zunächst auskommen. Aber: Es gibt Tricks...

\end{frame}
 
 
\begin{frame}
\frametitle{Wie funktionieren Sensoren?}
\begin{itemize}
\item Platzhalter für unsere Erörterungen!
\item Herr Schlenker macht dann Copy and Paste aus dem Chat
\end{itemize}
\end{frame}


\begin{frame}
\frametitle{Wer misst, misst Mist!}
Und welche Fails resultieren daraus?
\begin{itemize}
\item Flugzeug: Fahrwerk auf dem Boden eingezogen
\item \href{https://twitter.com/nke_ise/status/897756900753891328}{Desinfektionsmittelspender erkennt nur helle Haut}
\item Entfernungssensor erkennt keine Wollmäntel
\item \href{https://youtu.be/jI-rC2FCKo4?t=94}{Prellende Schalter}
\item Fallen Euch mehr Beispiele ein?
\end{itemize}
\end{frame}

 
 \begin{frame}
\frametitle{Was lernen wir daraus?}
Wir müssen die Funktionsweise jedes Sensors genau kennen, um Problemen begegnen zu können. 

Wir müssen für ,,Fallbacks'' sorgen!
\end{frame}
 
\begin{frame}
\frametitle{Loslegen mit Makecode}
\begin{itemize}
\item 
Die gesamte Programmierung findet im Browser statt 
\href{https://makecode.mindstorms.com/}{https://makecode.mindstorms.com/}
\item Fertige Programme werden als UF2-Datei heruntergeladen - sichert sie gut weg!
\item Makecode gibt es auch für \href{https://makecode.minecraft.com/}{Minecraft}!
\item ...und für die \href{https://arcade.makecode.com/}{Spieleprogrammierung}
\item ...und viele \href{https://makecode.adafruit.com/}{starke Microcontroller}
\end{itemize}
\end{frame}
 
\begin{frame}
\frametitle{Erstes Codebeispiel}
\begin{itemize}
\item Beim Start für 5 Sekunden ein schlafendes Gesicht anzeigen
\item Dann dauerhaft:
\item Erst zwei Sekunden nach rechts schauen
\item Dann zwei Sekunden nach links schauen
\end{itemize}
\end{frame}
 
\begin{frame}
\frametitle{Erweiterung um Events}
Wenn der Mittelknopf gedrückt wird, drei Sekunden ,,Dizzy'' anzeigen.

Seht Ihr das Problem? Könnt Ihr Euch weitere Probleme vorstellen?

\begin{figure}
  \includegraphics[width=8cm]{hic.jpg}
  \caption{HIC SVNT DRACONES}
  \label{fig:hic1}
\end{figure}


\end{frame}
 
\begin{frame}
\frametitle{Eine Variable hilft}
Das Problem: Events in separaten eckigen Blöcken \textbf{unterbrechen} entweder den Ablauf des forever()-Loops oder \textbf{laufen} parallel. Beides ist meist nicht erwünscht – \textbf{also vermeiden}! 

Emil zeigt in Minecraft, wie es besser geht: Event \textbf{setzt nur eine Variable}.

(hier kommt dann der Screenshot des Codes)

\end{frame}
 

\begin{frame}
\frametitle{Eine Variable für Notaus}
(hier kommt dann der Screenshot des Codes)
\end{frame}

\begin{frame}
\frametitle{Und eine Variable, um Code abzukürzen}
(hier kommt dann der Screenshot des Codes)
\end{frame}

\begin{frame}
\frametitle{Noch Zeit? Debugging-Ausgaben}
(hier kommt dann der Screenshot des Codes)
\end{frame}

\begin{frame}
\frametitle{Dumm, dass Corona Distanz erzwingt}
\begin{itemize}
\item Als Fernsessions wird Theorie einfacher als Praxis
\item Immerhin: Programmierumgebung hilft bei der Simulation
\item 3. Dezember 14:00 Roboterausgabe in Raum 2.108
\item 17. Dezember 15:30 Fernsession, evtl. 2er-Teams aus einer Klasse, wir programmieren einen Roboter, der Hindernissen ausweicht
\item Über die Weihnachtsferien: YouTube-Videos (auch von mir!) und eine kleine Programmieraufgabe, die keine Hardware erfordert
\end{itemize}



\end{frame}

% \begin{frame}
% \frametitle{Der Anspruch – der Schule und von uns selbst}
% \begin{itemize}
% \item  Wir lernen die Interaktion von Elektronik und physischer Welt
% \item  Unsere Arbeit ist eher ingenieurstechnisch als wissenschaftlich
% \item Physik und Technik: hier eher qualitativ als quantitativ
% \item  Wir machen Fehler und lernen daraus!
% \item Wir lernen auch aus Fehlern anderer – hilft schneller zu sein
% \item  Wir arbeiten gemeinsam auf Wettbewerbe hin
% \item  Jeder lernt alles: Programmieren und Konstruktion 
% \item  Bei Ad-Hoc-Wettbewerben dennoch Rollen im Team – Konstrukteur:innen, Programmierer:innen und Integrator:innen 
% \end{itemize}
% \end{frame}

% \begin{frame}
% \frametitle{Die Werkzeuge – und Umgang damit}
% \begin{itemize}
% \item  LEGO Technik zur Konstruktion
% \item  Mindstorm liefern autonome Steuereinheit, Sensoren, Aktoren
% \item Programmierung in \href{https://makecode.mindstorms.com/}{Makecode} (,,Klötzchen''), Python (Fortgeschrittene)
% \item  Wir lernen Dokumentation von Soft- und Hardware
% \item  Wir arbeiten strukturiert mit Versionierung \& Kommentierung!
% \item Wir benutzen ,,Rubber-Ducking'' zur Fehlersuche
% \item  Am Ende: Stand festhalten, Code sichern, Arbeitsplatz aufräumen!
% \item Materialien: Austausch LernSax, Folien und Beispielcode auf \href{https://github.com/mschlenker/GtaRoboter2020}{GitHub}
% \end{itemize}
% \end{frame}




\end{document}
