\documentclass{beamer}

\usepackage[utf8]{inputenc}
\usepackage{hyperref}
\hypersetup{
    colorlinks = true
    }
\usepackage{graphicx}
\usepackage{listings}

%Information to be included in the title page:
\title{Einführung in MicroPython und CircuitPython}
\author{Mattias Schlenker}
\institute{Wilhelm-Ostwald-Gymnasium}
\date{14. Juli 2021}

\begin{document}

\frame{\titlepage}

\begin{frame}
\frametitle{Herausforderung Hardware}
\begin{itemize}
\item Ziel war es, einen Python-Interpreter auf 32 Bit Microcontroller zu bringen
\item Microcontroller sind oft Fingernagel kleine Computer mit geringstem Energiebedarf
\item Einsatzgebiete: Steuerungs- und Regelungs-Aufgaben, Messwerte erfassen
\item Kosten: von unter 1 Euro bis ca. 15 Euro
\item Problem: Wenig Festspeicher (oft unter 1MB) und wenig Arbeitsspeicher 
\end{itemize}
\end{frame}

\begin{frame}
\frametitle{Die Lösung}

\begin{itemize}
\item ,,Baremetal Python'' – der Python-Interpreter läuft ohne Betriebssystem direkt auf dem Microcontroller
\item Abgespeckte Standardbibliothek
\end{itemize}

Ergebnis: Micropython läuft bereits auf 32 Bit Microcontrollern mit 32kB RAM und 256kB Festspeicher. Mehr ist natürlich immer besser…

\end{frame}


\begin{frame}
\frametitle{MicroPython oder CircuitPython?}

\begin{itemize}
\item MicroPython ist das ältere Projekt mit der breiteren Hardwareunterstützung
\item CircuitPython basiert auf Micropython, hat aber eine leichter verständliche Klassenbibliothek für Interaktion mit dem Microcontroller und ist intuitiver nutzbar
\end{itemize}

Empfehlung: Steht der Neukauf eines Microcontroller-Boards an, ein paar Euro mehr investieren und ein von CircuitPython unterstütztes Board kaufen, ist ein Board vorhanden, das nur MicroPython unterstützt, dann halt MicroPython.

\end{frame}

\begin{frame}
\frametitle{Installation am Beispiel Xiao (CircuitPython auf SAMD21)}

\begin{itemize}
\item Download der UF2-Datei von \href{https://circuitpython.org/}{circuitpython.org}
\item Reset-Pins zweimal hintereinander kurzschließen 
\item USB-Massenspeicher-Laufwerk wird angezeigt
\item UF2-Datei dorthin verschieben 
\item Microcontroller rebootet
\end{itemize}

Installation ist ohne zusätzliche Tools möglich. Programmierung auch. Dazu gleich…
\end{frame}

\begin{frame}
\frametitle{Installation am Beispiel ESP32 Devkit (MicroPython)}

\begin{itemize}
\item Download der BIN-Datei von \href{https://micropython.org/}{micropython.org}
\item Installation der Python-Programme esptool und ampy
\item esptool --chip esp32 --port COM4 write\_flash -z 0x1000 esp32….bin
\item Auf serieller Konsole zugreifen und mit ,,import webrepl\_setup'' die Webkonsole aktivieren
\item Microcontroller rebooten
\end{itemize}

Bei aktiviertem WebREPL sind keine zusätzlichen Tools nötig. Die Programmierung kann an jedem PC stattfinden. Dazu gleich...

\end{frame}

\begin{frame}
\frametitle{Programmierung am Beispiel Xiao (CircuitPython)}

\begin{itemize}
\item Beim Anschließen an den PC öffnet sich ein USB-Laufwerk
\item Hier kann mit jedem guten Editor die ,,main.py'' verändert werden
\item Speichert man die Datei, startet der Interpreter neu
\item Serielle Konsole kann bspw. mit PuTTY zugegriffen werden
\item Adafruit empfiehlt den Editor ,,\href{Mu}{https://codewith.mu/}''
\end{itemize}

\end{frame}

\begin{frame}[fragile]

\frametitle{Blink-Beispiel CircuitPython}

\begin{lstlisting}[language=Python] 
import board
import digitalio
import time

led = digitalio.DigitalInOut(board.LED)
led.direction = digitalio.Direction.OUTPUT

while True:
    led.value = True
    time.sleep(0.5)
    led.value = False
    time.sleep(0.5)

\end{lstlisting} 

(Beispiel mit Auslesen eines Pins wird vorgeführt)

\end{frame}

\begin{frame}[fragile]

\frametitle{Blink-Beispiel MicroPython}

In Mu speichern und per WebREPL hochladen...

\begin{lstlisting}[language=Python] 
import machine
import time

led = machine.Pin(5, machine.Pin.OUT)

while True:
    led.on()
    time.sleep(0.5)
    led.off()
    time.sleep(0.5)

\end{lstlisting} 

Achtung: Pin-Mapping entspricht oft nicht den Aufdrucken am Board. Pinout-Diagramme heranziehen!

\end{frame}

\begin{frame}[fragile]
\frametitle{Links und Fazit}

\begin{itemize}
\item Ich habe 3 ESP ausprobiert, bis ich einen gefunden habe, bei dem serielle Konsole und WIFI zuverlässig funktionierten
\item CircuitPython ist von Handhabung und Stabilität zu bevorzugen
\item \href{https://circuitpython.org}{CircuitPython}
\item \href{https://micropython.org}{MicroPython}
\item \href{http://micropython.org/webrepl/}{WebREPL}
\item \href{https://codewith.mu/}{EditorMu}
\item \href{https://github.com/mschlenker/GtaRoboter2020/tree/master/Einzelveranstaltung\_Circuitpython}{https://github.com/mschlenker/GtaRoboter2020 - Einzelveranstalung CircuitPython - diese Folien}
\end{itemize}

\end{frame}

\end{document}
